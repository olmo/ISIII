\documentclass[a4paper,12pt,oneside]{book}
\usepackage[utf8]{inputenc}
\usepackage[spanish]{babel}
\usepackage{graphicx}
\usepackage[pdftex]{color}
\usepackage{ucs}
\usepackage{listings}
\usepackage{amsfonts}
\usepackage{amssymb}
\usepackage{hyperref}
\usepackage{url}
\usepackage{multirow}
\usepackage{listings}  
\usepackage{color} 
\usepackage{colortbl}
\usepackage{phdthesis}
\usepackage{amsmath}
\usepackage[all]{xy}
\usepackage{times}
\usepackage{setspace}
%\onehalfspacing
\usepackage{epstopdf}
\usepackage{float}
\usepackage{hyperref}
\onehalfspacing
\renewcommand{\baselinestretch}{1.5}
% \doublespacing

% opening
\title{blablablabla}
\author{blablabla \\
  blebleble}
\date{Curso 2011/12}
\parskip = 7 pt

\makeindex

\begin{document}

\tableofcontents

\pagebreak

\section{Objetivo}

\section{Alcance}

\section{Recursos}

\subsection{Recursos humanos}
Los 12 componentes del equipo.

\subsection{Recursos software}
\subsubsection{Github}

\subsubsection{SWAD}
Se hará uso de esta herramienta para que los profesores puedan recibir nuestro trabajo del proyecto 
de forma actualizada con respecto a lo que tendremos en nuestro repositorio.

\subsubsection{Google Docs}
Será utilizado para poder hacer borradores de forma rápida y sencilla a tiempo real entre los diferentes miembros
del grupo

\subsubsection{Skype}
Útil para poder llevar a cabo reuniones y trabajos entre los miembros del grupo sin tener que quedar físicamente
en un lugar facilitando y flexibilizando el horario de todos los componentes.

\subsubsection{Eclipse}
IDE de desarrollo potente y libre. Lo usaremos debido a que tiene muchas extensiones que nos facilitarán el trabajo
a la hora de diseñar y desarrollar. Por ejemplo la herramienta de UML se puede instalar como una extensión de eclipse.

\subsubsection{SDK Android}

\subsubsection{Latex}

\subsubsection{UMLET}
Además de tratarse de una herramienta de software libre es suficientemente potente para nuestro trabajo. Aparte de poder
instalarse como una aplicación posee la ventaja de poder integrarse como una extensión del IDE eclipse que vamos a usar

\subsubsection{Planificación}

\subsection{Recursos hardware}

\section{Organización del equipo}

\section{Tareas}


\end{document}
