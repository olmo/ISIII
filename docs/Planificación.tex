\documentclass[a4paper,12pt,oneside]{book}
\usepackage[utf8]{inputenc}
\usepackage[spanish]{babel}
\usepackage{graphicx}
\usepackage[pdftex]{color}
\usepackage{ucs}
\usepackage{listings}
\usepackage{amsfonts}
\usepackage{amssymb}
\usepackage{hyperref}
\usepackage{url}
\usepackage{multirow}
\usepackage{listings}  
\usepackage{color} 
\usepackage{colortbl}
\usepackage{phdthesis}
\usepackage{amsmath}
\usepackage[all]{xy}
\usepackage{times}
\usepackage{setspace}
%\onehalfspacing
\usepackage{epstopdf}
\usepackage{float}
\usepackage{hyperref}
\onehalfspacing
\renewcommand{\baselinestretch}{1.5}
% \doublespacing

% opening
\title{blablablabla}
\author{blablabla \\
  blebleble}
\date{Curso 2011/12}
\parskip = 7 pt

\makeindex

\begin{document}

\tableofcontents

\pagebreak

\section{Objetivo del proyecto}
El objetivo del proyecto es realizar un programa para la gestión de una asociación de ayuda a colectivos desfavorecidos. 
La aplicación debe gestionar los socios de la asociación así como las aportaciones de los socios, colaboradores y empresas.

También debe gestionar los beneficiarios de la asociación y las ayudas aportadas a estos, además de una bolsa de trabajo para los beneficiarios.

Se realizará una segunda aplicación para una plataforma móvil que permitirá a los socios gestionar sus datos y consultar sus aportaciones a la asociación. 
\section{Declaración del alcance}
Se desea modelar un producto software para llevar la gestión de la ONG Diaketas, encargada de la atención a los colectivos más 
desfavorecidos de nuestra sociedad. Realizan su labor gracias a las aportaciones de socios, colaboradores, empresas y/o anónimos.

La ONG obtiene aportaciones de varios tipos de donantes:
\begin{itemize}
	\item Socio: Realiza aportaciones periódicas, eligiendo el plazo y con posibilidad de modificarlo en el futuro. En caso de no realizar un pago en el plazo elegido, este quedará pendiente hasta que se dé de baja o lo haga.
	\item Colaborador: Realiza aportaciones puntuales.
	\item Empresa/Anónimo: Realiza aportaciones tanto puntuales como periódicas. En caso de ser anónimo no se guardará información sobre su identidad.
\end{itemize}


Los trabajadores de la ONG accederán a la aplicación de escritorio identificándose con su nombre de usuario y contraseña.

A través de la aplicación de escritorio los trabajadores podrán realizar lo siguiente:

\begin{itemize}
	\item Gestión de usuarios del sistema, creándolos, borrándolos o modificándolos.
	\item Gestión de todos los tipos de información necesarios para trabajadores, donantes y beneficiarios mediante altas, bajas, modificaciones y consultas.
	\item Gestión de todos los tipos de donaciones realizadas por los socios (monetaria, alimentos, materiales, voluntariado,etc...) mediante inserción, borrado, modificación
	consulta y listados periódicos de las donaciones recibidas.
	\item Gestión de las ayudas dadas a los beneficiarios para controlarlas de una forma fácil.
	\item Gestión de la bolsa de empleo privada a través de ofertas de trabajo de empresas y demandas de empleo de los beneficiarios.
\end{itemize}

Todas las gestiones anteriormente citadas serán accesibles de forma intuitiva para usuarios con conocimientos limitados de informática.

Sobre los datos que contendrá nuestro sistema:
\begin{itemize}
	\item De los trabajadores guardaremos su información personal, su nombre de usuario y contraseña para el acceso a la aplicación.
	\item Sobre los beneficiarios tendremos almacenada la información solicitada en la ficha anexa. Si no se entrega el certificado de empadronamiento se creará un registro temporal, si tras un periodo razonable no se ha entregado dicho certificado el trabajador eliminará al beneficiario.
	\item Sobre los donantes se almacenarán los datos personales a excepción de los anónimos. También guardaremos la forma de pago, la periodicidad en los casos necesarios y certificado de donaciones IPRF para las empresas.
\end{itemize}

Toda la información anterior quedará almacenada durante un año tras darse de baja el usuario. Si éste pide explícitamente el borrado de su información, se llevará a cabo de forma inmediata.

\subsection{Objetivos}

\subsection{Funciones}

\subsection{Rendimiento}

\subsection{Fiabilidad}

\subsection{Interfaces}

\subsection{Restricciones}


\section{Determinación de recursos}

\subsection{Recursos humanos}
Los 12 componentes del equipo.

\subsection{Recursos software}
\subsubsection{Github}
Github es un servicio online de almacenamiento de proyectos basado en sistema de control de versiones git. Es un servicio gratuito para proyectos de código abierto que utilizan un gran número de proyectos de código abierto y de software libre.

Lo hemos escogido porque tiene las herramientas necesarias para suplir nuestras necesidades y así no necesitamos montar un servidor propio. Entre las herramientas que ofrece:
\begin{itemize}
	\item Repositorio para Git: un sistema gestor de versiones ampliamente utilizado en la actualidad.
	\item Sistema de seguimiento de tareas.
	\item Wiki.
	\item Revisión de cambios en el código.
\end{itemize}

\subsubsection{SWAD}
Se hará uso de esta herramienta para que los profesores puedan recibir nuestro trabajo del proyecto 
de forma actualizada con respecto a lo que tendremos en nuestro repositorio.

\subsubsection{Google Docs}
Será utilizado para poder hacer borradores de forma rápida y sencilla a tiempo real entre los diferentes miembros
del grupo

\subsubsection{Skype}
Útil para poder llevar a cabo reuniones y trabajos entre los miembros del grupo sin tener que quedar físicamente
en un lugar facilitando y flexibilizando el horario de todos los componentes.

\subsubsection{Eclipse}
IDE de desarrollo potente y libre. Lo usaremos debido a que tiene muchas extensiones que nos facilitarán el trabajo
a la hora de diseñar y desarrollar. Por ejemplo la herramienta de UML se puede instalar como una extensión de eclipse.

\subsubsection{SDK Android}
Vamos a utilizar el SDK de Android para realizar la aplicación móvil para la plataforma android. Hemos decidido utilizarlo porque es un sistema operativo de código abierto del que ya disponen una gran variedad de móviles y las aplicaciones se programan en java, al igual que la aplicación principal de nuestro proyecto.

Utilizaremos la versión 10 de la api porque es la versión que utilizan la mayoría de los móviles por el momento. Así nos aseguramos una mayor compatibilidad.

\subsubsection{\LaTeX{}}
Los documentos estarán escritos en \LaTeX{} para que junto al sistema gestor de versiones puedan trabajar simultáneamente varias personas sobre el mismo documento y luego se incorporen los cambios automáticamente. Evita que una persona tenga que encargarse de maquetar el fichero final.

\subsubsection{UMLET}
Además de tratarse de una herramienta de software libre es suficientemente potente para nuestro trabajo. Aparte de poder
instalarse como una aplicación posee la ventaja de poder integrarse como una extensión del IDE eclipse que vamos a usar

\subsubsection{GanttProject}
Con este software generaremos los diagramas de gantt para las tareas del proyecto. Lo hemos elegido por ser software libre y multiplataforma, y porque guarda los proyectos en xml lo que facilita su sincronización con el sistema de gestión de versiones.

\subsection{Recursos hardware}
\begin{itemize}
	\item PC personal * 12
	\item Dispositivo android
\end{itemize}

\section{Planificación organizativa}
Coordinador del equipo de desarrollo: Olmo Jiménez Alaminos

\subsubsection{Subgrupo A}
\begin{itemize}
	\item Ángel Costela Sanmiguel
	\item Olmo Jiménez Alaminos
	\item Jesús Linares Bolaños (Coordinador)
	\item Miguel Cantón Cortés
\end{itemize}

\subsubsection{Subgrupo B}
\begin{itemize}
	\item Víctor Cabezas Lucena (Coordinador)
	\item Pedro Sánchez de Castro
	\item David Medina Godoy
	\item Alejandro Merlo Serrano
\end{itemize}

\subsubsection{Subgrupo C}
\begin{itemize}
	\item Pablo Calvo Cabezas
	\item Javier Castillo Carmona
	\item Juan Manuel Lucena Morales(Coordinador)
	\item Miguel Morales Rodríguez
\end{itemize}


\begin{tabular}{c c c c}
\hline\hline
 & Planificación & Diseño & Implementación \\ [0.5ex]
\hline
Iteración 1 & A & B & C \\
Iteración 2 & B & C & A \\
Iteración 3 & C & A & B \\ [0.5ex]
\hline
\end{tabular}


\section{Planificación temporal}


\end{document}
