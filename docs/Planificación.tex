\documentclass[a4paper,12pt,oneside]{book}
\usepackage[utf8]{inputenc}
\usepackage[spanish]{babel}
\usepackage{graphicx}
\usepackage[pdftex]{color}
\usepackage{ucs}
\usepackage{listings}
\usepackage{amsfonts}
\usepackage{amssymb}
\usepackage{hyperref}
\usepackage{url}
\usepackage{multirow}
\usepackage{listings}  
\usepackage{color} 
\usepackage{colortbl}
\usepackage{phdthesis}
\usepackage{amsmath}
\usepackage[all]{xy}
\usepackage{times}
\usepackage{setspace}
%\onehalfspacing
\usepackage{epstopdf}
\usepackage{float}
\usepackage{hyperref}
\onehalfspacing
\renewcommand{\baselinestretch}{1.5}
% \doublespacing

% opening
\title{blablablabla}
\author{blablabla \\
  blebleble}
\date{Curso 2011/12}
\parskip = 7 pt

\makeindex

\begin{document}

\tableofcontents

\pagebreak

\section{Objetivo}

\section{Alcance}

\section{Recursos}

\subsection{Recursos humanos}
Los 12 componentes del equipo.

\subsection{Recursos software}
\subsubsection{Github}
Github es un servicio online de almacenamiento de proyectos basado en sistema de control de versiones git. Es un servicio gratuito para proyectos de código abierto que utilizan un gran número de proyectos de código abierto y de software libre.

Lo hemos escogido porque tiene las herramientas necesarias para suplir nuestras necesidades y así no necesitamos montar un servidor propio. Entre las herramientas que ofrece:
\begin{itemize}
	\item Repositorio para Git: un sistema gestor de versiones ampliamente utilizado en la actualidad.
	\item Sistema de seguimiento de tareas.
	\item Wiki.
	\item Revisión de cambios en el código.
\end{itemize}

\subsubsection{SWAD}

\subsubsection{Google Docs}

\subsubsection{Skype}

\subsubsection{Eclipse}

\subsubsection{SDK Android}
Vamos a utilizar el SDK de Android para realizar la aplicación móvil para la plataforma android. Hemos decidido utilizarlo porque es un sistema operativo de código abierto del que ya disponen una gran variedad de móviles y las aplicaciones se programan en java, al igual que la aplicación principal de nuestro proyecto.

Utilizaremos la versión 10 de la api porque es la versión que utilizan la mayoría de los móviles por el momento. Así nos aseguramos una mayor compatibilidad.

\subsubsection{\LaTeX{}}
Los documentos estarán escritos en \LaTeX{} para que junto al sistema gestor de versiones puedan trabajar simultáneamente varias personas sobre el mismo documento y luego se incorporen los cambios automáticamente. Evita que una persona tenga que encargarse de maquetar el fichero final.

\subsubsection{UML}

\subsubsection{GanttProject}
Con este software generaremos los diagramas de gantt para las tareas del proyecto. Lo hemos elegido por ser software libre y multiplataforma, y porque guarda los proyectos en xml lo que facilita su sincronización con el sistema de gestión de versiones.

\subsection{Recursos hardware}

\section{Organización del equipo}

\section{Tareas}


\end{document}
